%-------------------------------------------------------------
% Structure
%-------------------------------------------------------------

% Reset Page and Section numbers
\newcommand{\setNumbers}[2]{
	\renewcommand{\thesection}{#2{chapter}.#2{section}}
	\renewcommand{\theHsection}{#2{chapter}.#2{section}}
	\setcounter{chapter}{0}
	\setcounter{section}{0}
	\pagenumbering{#1}
	\setcounter{page}{1}
}

%-------------------------------------------------------------
% Figures
%-------------------------------------------------------------

\newcommand{\fig}[3]{
\begin{figure}[h]
\vspace{1em}
	\centering
	\includegraphics[width=.9\linewidth]{#1}
	\caption{#3}
	\label{fig:#2}
\vspace{1em}
\end{figure}
}

\newcommand{\uml}[3]{
\begin{figure}
		\centering
		\includesvg{#1}
		\caption{#3}
		\label{fig:#2}
\vspace{1em}
\end{figure}
}

%-------------------------------------------------------------
% Quotes
%-------------------------------------------------------------

% A direct quote
% @par1: The quoted text
% @par2: The source where the text is from
% @par3: The page where the text is from
\newcommand*{\QuoteDirect}[3]{\QuoteM{\emph{#1}} \cite[#3]{#2}}

% A direct quote without page
% @par1: The quoted text
% @par2: The source where the text is from
\newcommand*{\QuoteDirectNoPage}[2]{\QuoteM{\emph{#1}} \cite{#2}}

% A indirect quote
% @par1: The source where the text is from
% @par2: The page where the text is from
\newcommand*{\QuoteIndirect}[2]{(vgl. \cite[#2]{#1})}

% A indirect quote without page
% @par1: The source where the text is from
\newcommand*{\QuoteIndirectNoPage}[1]{(vgl. \cite{#1})}

% A text with quotation marks
% @par1: The text you want to quote
% »text«
\newcommand*{\QuoteM}[1]{\frqq #1\flqq}

% A text with single quotation marks
% @par1: The text you want to quote
% ›text‹
\newcommand*{\QuoteMs}[1]{\frq #1\flq}

% To adjust some words to the flow
% @par1: the adjusted words
% [text]
\newcommand*{\AdjustWords}[1]{{\normalfont[#1]}}

% Displays a reference to the given object
% @par1: the lable of the thing you want to see
% (Siehe auch Abbildung 1.1 »Ein Bild« auf Seite 4)
\newcommand*{\SeeS}[1]
{(siehe \autoref{#1} \QuoteM{\nameref{#1}})}

% Displays a reference to the given object
% @par1: the lable of the thing you want to see
% (Siehe auch Abbildung 1.1 »Ein Bild« auf Seite 4)
\newcommand*{\SeeB}[1]
{(Siehe auch \autoref{#1} \QuoteM{\nameref{#1}} auf \autopageref{#1})}

% Displays a reference to an equation
% @par1: the lable of the equation you want to see
% (Siehe auch Gleichung 1.1 in »Dummy Section« auf Seite 5)
\newcommand*{\SeeEq}[1]
{(siehe auch \autoref{#1} in \QuoteM{\nameref{#1}} auf \autopageref{#1})}

% The symbol for a elision
% Is used for more than missings one word or a sentence
% [...]
\newcommand*{\Elision}{{\normalfont[\dots]}}

% The symbol for a small elision
% Is used for only one missing word
% [..]
\newcommand*{\ElisionSmall}{{\normalfont[..]}}

% This is used if a book is cited at whole
% passim means something like continuous
% text (vgl. [Aut99, passim]).
\newcommand*{\passim}{passim}

% To show the audience that there is something
% To display a wrong/importen part but not corrected in the quote
% text error [sic!] text
\newcommand*{\SIC}{{\normalfont[sic!]}}

% The text for a note from the author
% text, Anm. d. Autors
\newcommand*{\NoteFromAuthor}{{\normalfont\unskip , Anm. d. Autors}}
